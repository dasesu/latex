\documentclass{beamer}

\usepackage[utf8]{inputenc} % Para que reconozca la codificacion
\usepackage[spanish,es-tabla]{babel} % Para idioma en español
\usepackage[T1]{fontenc} % Para Codificación tipo 1 del archivo.
\usepackage{xcolor} % Para cambiar el color de las letras
\usepackage{pagecolor} % Para cambiar el color de las pagina
\usepackage{colortbl} % PAra usar colores en tablas
\usepackage{graphicx} % Para incluir graficos
\usepackage{placeins} % para evitar que se descuadre el texto con respecto a las imagenes
\usepackage{wrapfig} %  Para texto alrededor de las figuras
\usepackage{amsmath} % Para más símbolos matemáticos
\usepackage{amsthm} % Para más símbolos matemáticos
\usepackage{amssymb} % Para incluir algunos simbolos matematicos
\usepackage{marvosym} % Para incluir simbolo de Euro
\usepackage{fontawesome} % 
\usepackage{pifont} % Para incluir simbolos ding
\usepackage{array}
\usepackage{multirow} % Permite construir tablas en la que las celdas ocupan varias filas 
\usepackage{dcolumn} % Para dividir columna por una diagonal
\usepackage{parskip} % Descomentar para que los parrafos no comiencen con sangria. y espacio interlineado
\usepackage{tikz} % Para framezoom con circulos
\usepackage{soul} % Para framezoom con circulos
\usepackage{hyperref}

\usetikzlibrary{spy} % Para framezoom con circulos
\setbeamercovered{transparent} % para que los items se vean en transparente antes de ser seleccioandos

\usepackage{enumerate} %Para trabajar con listas enumeradas
\usepackage{multimedia} % para incluir videos
\usepackage{background}

\usetheme{lucid}



\setbeamertemplate{background}{\tikz[overlay,remember picture]\node[opacity=0.2]at (current page.center){\includegraphics[width=16cm]{img/texturafondo.jpg}};}

\institute{Nombre de la Institucion \\ \small{Información adicional} }
\title{Título de la presentación}
\subtitle{Un subtítulo}
\author{Autor}
%\date{}

\AtEndDocument{\usebeamertemplate{endpage}}

\begin{document}

% ------------------------------------ % 

   % Pagina 1
   \begin{frame}[plain]
   \titlepage % como colocar el titulo de la primera pagina
   \end{frame}

% ------------------------------------ % 

   % Pagina 2
   \begin{frame}
   \frametitle{Tabla de Contenido} % una forma de colocar el titutlo
   \tableofcontents % indice de contenido
   \end{frame}

% ------------------------------------ % 

   % Pagina 3
   \section{Marco Teórico}  % definiendo secciones y subsecciones, esto es lo que en realidad va en el indice
   \begin{frame}{Some title} %otra forma de colocar el titulo
   \framesubtitle{Some subtitle}
      \begin{itemize}
         \item Beamer is a \LaTeX class that allows you to create presentations
         \item The project home page is \url{http://latex-beamer.sourceforge.net/}
         \item Beamer contains several themes, but they are a bit ugly
          \begin{itemize}
             \item But with a lot of useful features, such as navigation bars, outlines, etc.
          \end{itemize}
         \item Torino is a pretty theme
          \begin{itemize}
             \item With a lot of useless - but pretty - features
             \item But without some useful features
             \item Well suited for short talks, for longer talks you should use themes with navigation bars
          \end{itemize}
          \item Why the name?
      \end{itemize}
   \end{frame}

% ----------------------------------------- %


  \begin{frame}
    \frametitle{Un bloque natural} 
    \begin{block}{Título del bloque}
    \begin{itemize}
      \item Elemento
      \item Otro elemento
    \end{itemize}
    \end{block}
  \end{frame}

% ----------------------------------------- %

   \begin{frame}
   \frametitle{Mas ejemplos de bloques} 
   \begin{exampleblock}{Título del bloque}<+->
      \begin{itemize}
      \item Un item
      \item Otro item
      \item Otro item más
      \end{itemize}
   \end{exampleblock}

   \begin{exampleblock}{Título del bloque}<+->
      \begin{itemize}
      \item<+-> Un item
      \item<+-> Otro item
      \item<+-> Otro item más
      \end{itemize}
   \end{exampleblock}
   \end{frame}

% ----------------------------------------- %

  \begin{frame}
    \frametitle{Un blocque de Alerta} 
    \begin{alertblock}{Título del bloque}
      \begin{itemize}
        \item Un item
        \item Otro item
        \item Otro item más
      \end{itemize}
    \end{alertblock}
  \end{frame}
% ------------------------------------ % 

   \begin{frame}
   \frametitle{Title}
   \framesubtitle{subtitle}
     \begin{itemize}
       \item<+-> This is on the first only
       \item<+-> This is on the first three slides
       \item<+-> This is on the second to fourth slides and the sixth slide
     \end{itemize}
   \end{frame}




\begin{frame}{Referencias}

\bibliographystyle{IEEEtran.bst} % estilo de la bibliografía.
\bibliography{partes/bibliografia} % yyyy.bib es el fichero donde está salvada la bibliografía.
\end{frame}

\end{document}
