\chapter{Capítulo de Ejemplo}

Llamada a pié de página linkeable \footnote{Texto de la nota al pié}\par
Ejemplo de cita bibliográfica \cite{nordquist2020basic}\par
Ejemplo de cita bibliográfica indicando el número de página \cite[pág 21]{stokoe2005sign}\par
\lipsum[1-2]

\section{Sub Sección}
   \lipsum[1-2]

\subsection{Sub-sub Sección}
   \lipsum[1]
   \begin{itemize}
      \item Ejemplo de item de primer nivel
      \item Ejemplo de item de primer nivel 
      \begin{itemize}
         \item Ejemplo de item de segundo nivel
         \item Ejemplo de item de segundo nivel
         \begin{itemize}
            \item Ejemplo de item de tercer nivel
            \item Ejemplo de item de tercer nivel
         \end{itemize}
      \end{itemize}
      \item Ejemplo de item de primer nivel 
   \end{itemize}

   \lipsum[1-2]


      \begin{figure}[h]
         \centering
         \includegraphics[scale=0.5]{350x150.png}
         \caption[Imagen Ejemplo]{Imagen de rectangulo gris de 350x150px.}
         \label{imagendesc1}
         % Label: es para realizar referencias cruzadas,
         % notar que se escribe todo junto y sin acentos.
      \end{figure}


      \begin{table}[h]
      \begin{center} % tabla centrada en el texto
      \caption{Tabla de ejemplo.}
         \begin{tabular}{l c} %l -> left, c -> center, r -> right
            \hline % línea horizontal
            \hline
            Título 1 & Título 2 \\  % Fila de encabezados
            \hline
            Cuadro 1 & Cuadro 2 \\  % (\\ indica el final de la línea)
            Cuadro 3 & Cuadro 4 \\
            \hline
            \hline
         \end{tabular}
         \label{tablaUno}
      \end{center}
      \end{table}


      \section{Ejemplo de titulo 2}
         \lipsum[1]
      \subsection{Ejemplo de titulo 3}
            \lipsum[1]
      \subsubsection{Ejemplo de titulo 4 (no numerado)}
            \lipsum[1]
      \paragraph{Ejemplo de título 5: (no numerado)}
               \lipsum[1]

      Este texto apunta a una referencia \ref{imagendesc}

      \section{Discusión de resultados}
         \lipsum[1-2]
         \begin{figure}[h]
            \centering
            \includegraphics[scale=0.5]{350x150.png}
            \caption{Descripción de la imagen}
            \label{imagendesc}
            % Label: es para realizar referencias cruzadas,
            % notar que se escribe todo junto y sin acentos.
         \end{figure}
