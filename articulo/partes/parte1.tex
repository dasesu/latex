\section{Introducción}

Ejemplo de llamada a pie de pagina "linkeable" \footnote{Texto de la nota al pie}\par
Ejemplo 1 de cita bibliográfica \cite{tapscott2016blockchain}\par
Ejemplo 2 de cita bibliográfica\cite[pág 21]{tapscott2016blockchain}\par
\lipsum[1-2]

\subsection{Sub Título}
   \lipsum[1-2]

\subsubsection{Sub sub título}
   \lipsum[1]
   \begin{itemize}
      %[topsep=0pt,itemsep=-1ex,partopsep=1ex,parsep=1ex]
      \item Si el contenido de una bala excede una línea, entonces todas las líneas deben empezar
            en el mismo lugar, es decir que deben tener la misma indentación como lo muestra esta
            bala.
      \item Antes de empezar una lista de balas de segundo nivel, es bueno dejar un espacio vertical
            pequeño (aquí se dejó un espacio de 6 puntos) entre la bala de primer nivel y la primera
            bala de segundo nive
      \begin{itemize}
         \item Las balas de segundo nivel se podrían identificar por el símbolo de punta de flecha.
               Lo importante es que sean diferentes a las balas de primer nivel.
         \item Deben existir varios ítems en la lista de balas. Es decir, una lista de balas con un
               único elemento no tiene razón de ser.
         \begin{itemize}
            \item Al finalizar una lista de balas de segundo nivel, es bueno dejar un espacio vertical
                  pequeño (aquí se dejó un espacio de 3 puntos) antes de la siguiente bala de primer nivel.
            \item Al finalizar una lista de balas de segundo nivel, es bueno dejar un espacio vertical
                  pequeño (aquí se dejó un espacio de 3 puntos) antes de la siguiente bala de primer nivel.
         \end{itemize}
      \end{itemize}
      \item La indentación de las balas es importante. Cuando la indentación es muy grande, no se
            ve bien la lista de balas
   \end{itemize}
   \lipsum[1-2]
   %\cite{solminihac2005}
   %ejemplo de cita 'En el texto' sin parentesis, para citar con parentesis usar \citep{}

      \begin{table}[h]
      \begin{center} % tabla centrada en el texto
         \begin{tabular}{l c} %l -> left, c -> center, r -> right
            \hline % línea horizontal
            \hline
            Título 1 & Título 2 \\  % Fila de encabezados
            \hline
            Cuadro 1 & Cuadro 2 \\  % (\\ indica el final de la línea)
            Cuadro 3 & Cuadro 4 \\
            \hline
            \hline
         \end{tabular}
         \label{tablaUno}
      \end{center}
      \caption{Título de la tabla.}
      \end{table}


         \begin{table}[h]
            \begin{center} % tabla centrada en el texto
               \begin{tabular}{|c|c|}
                  \hline
                  \rowcolor[gray]{0.9} \makebox[2.5cm][c]{\textbf{País}} & \textbf{Extensión}\\
                  \hline
                  celda 3 &\makebox[2.5cm][c]{celda 4}\\
                  \hline
                  celda 5 &\makebox[2.5cm][r]{celda 6}\\
                  \hline
               \end{tabular}
               \label{tablaUno}
            \end{center}
            \caption{Título de la tabla.}
         \end{table}

      
      \begin{figure}[h]
         \centering
         \includegraphics[scale=0.5]{350x150.png}
         \caption[short title]{Long caption describing the figure.}
         \label{imagendesc1}
         % Label: es para realizar referencias cruzadas,
         % notar que se escribe sin espacios ni acentos.
      \end{figure}
      \FloatBarrier


      \section{Ejemplo de titulo 2}
         \lipsum[1]
         \subsection{Ejemplo de titulo 3}
            \lipsum[1]
         \subsubsection{Ejemplo de titulo 4 (no numerado)}
            \lipsum[1]
            \paragraph{Ejemplo de título 5: (no numerado)}
               \lipsum[1]

      Este texto apunta a una referencia \ref{imagendesc}

      \section{Discusión de resultados}
         \lipsum[1-2]
         \begin{figure}[h]
            \centering
            \includegraphics[scale=0.5]{350x150.png}
            \caption{Descripción de la imagen}
            \label{imagendesc}
            % Label: es para realizar referencias cruzadas,
         \end{figure}


